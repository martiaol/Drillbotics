SPE's sub-committee DSATS (Drilling System Automation Technical Section) has organized a student competition in an effort to accelerate the uptake of automation in the drilling industry. The competition is called Drillbotics \texttrademark and this report is NTNU's contribution. 

The drill pipe and the drill bit will be provided by DSATS and the dimensions have been provided in the guidelines. The drill pipe is made of aluminium, has a length of 914 mm, and, an outer diameter of 9.53 mm and an inner diameter of 7.75 mm. The bit is a polycrystalline diamond compact (PDC) bit and has an outer diameter of 28.6 mm. The properties of the formation rock are unknown, but the dimensions are stated and are 30cmx30cmx60cm. The total maximum length of the stabilizers is limited to 90 mm. The total power consumption is limited to 25 hp and the weight on bit (WOB) is unlimited.

The main objective of the competition is to design a fully automated drilling rig that can autonomously drill a vertical well as quickly as possible while maintaining rig and drill string integrity. The purpose of this report is to present the team's solution to the problem and describe the rig's main design features.

The proposed rig design has been developed based on new ideas and innovative solutions, as well as current industry practices and guidelines provided by DSATS. Through research and analysis, an evaluation of the most likely drilling related problems and dysfunctions has been made and this has provided the basis for the dimensioning of the drilling machine and the architecture of the control system.

One of the main issues addressed in the report is how to ensure sufficient WOB without causing pipe failure. The idea is to solve this problem by adding a nozzle in the pipe which will increase its internal pressure and reduce the risk of failure. 

Another important design feature is how the system responds to drilling dysfunctions. Using an optimization function, the control system will use input data from downhole and surface sensors to adjust the control parameters. One of the most important measurements is downhole string vibrations. Large vibrations result in loss of energy and pipe fatigue and the optimization function will therefore seek to minimize their amplitude. In addition to this, key drilling parameters such as torque and ROP will be monitored and included in the optimization function to ensure high drilling efficiency.