One of the main concerns during any operation, regardless of its scale or scope, is the health and safety of the personnel. By automating the drilling process, the need for manual intervention is reduced and this in turn results in a safer 
drilling operation. The improved safety, as well as the increased efficiency and reduced costs, are some of the main advantages of automating drilling processes.

Although manual intervention is not required for a normal drilling operation, the construction, installation and transportation of the rig require physical labor which increases the risk of hazardous situations. Any unplanned 
incidents requiring manual intervention during drilling must also be considered.

Most accidents on a full-scale platform occur on the drill floor, especially during pipe handling \cite{drillcon}. Due to the scale of the rigs in this competition, this will not be an issue, so it is important to bear in mind that if this design were to be used for a full-scale operation, the safety hazards would be of a different nature.

The main safety hazards during the construction and operation of this drilling rig have been identified and safety 
precautions have been proposed in the sections below.

\subsection{Safety Hazards during Construction}
\subsubsection{Rig Construction}

During the period of construction of the rig, the hazardous situations will mostly be mechanical. The handling of equipment, material and debris will be the main concern.

To minimize the risk of any accidents occurring, safety glasses and protective footwear should be worn. Hearing protection, gloves and coveralls may also be required during construction activity. The construction area will be kept off limits to people not involved in the project.

As mentioned in the introduction, because of the small scale of the rig and the light weight of the equipment, the risk of any severe damage occurring is very small.


\subsubsection{Electrical Practices}
Safety related work practices will be used during the construction phase to prevent injuries resulting from electrical shock. No power will be supplied while connecting wires and components. All electrical connections will be secured and wiring will be insulated. Qualified personnel will be responsible for the high voltage setup in to the rig, while low voltage will be arranged by team members \cite{mtu}.

\subsection{Storage and Maintenance}
Any chemicals used will be documented. Disposal of waste must be controlled and done as stated in regulations.


\subsection{Safety During Transportation}
The rig is designed to ensure safe transportation. The derrick will be folded down to increase maneuverability and stability by lowering the center of mass. Jack-up casters are put on each leg to avoid heavy lifts for personnel and ensure easy rolling movement of the rig. 


\subsection{Safety Hazards During Operation}
\subsubsection{Unloading and Handling of Rock Sample}
The rig is designed to operate with the rock sample placed on the ground. This means that the rig will be maneuvered in place, instead of moving the rock sample around. No heavy lifting during rig up/down and operation will be needed, thereby reducing the risk of personnel injury and equipment damage.  


\subsubsection{Electrical System}
The water and the electrical system will be separated to avoid shorting and electrical hazards. All electrical components will be stored inside waterproof housing. Electrical cables will follow the motion of the carriage and will be exposed to wear because of bending and twisting. The state of the cables will continuously be monitored and changed if necessary. 


\subsubsection{Safety Factors and Dimensioning}
Safety factors will be applied to all calculations regarding failure of equipment. The motors and the pump will be dimensioned to fit the required criteria, such as operating RPM intervals and pump rate. The selected pump for the system has a higher pressure rating than required during the operation, a safety valve will therefore be included to ensure that the pressure does not exceed the critical values of the system. 


\subsubsection{Control System}
The control system and algorithm obtained will ensure the system is operating within safe intervals for weight on bit, torque, pressures and rotational speeds. These values have been implemented to avoid failure due to buckling, twist-off, burst and vibrations. Safety has the highest priority in the control system in which operating parameters will continuously be checked to be within safe range. If the parameters are outside the safety interval, the system will respond accordingly. Depending on how critical the values are, the algorithm will reduce RPM and WOB, lift the bit of bottom or shut down. 


\subsubsection{Emergency Shutdown of System}
Although the drilling process is fully automated, uncontrolled situations can occur. A manual stop button will cut the power supply to the motors and pump in case of emergency. The motion of the ball screw will stop immediately and keep the carriage in place. 