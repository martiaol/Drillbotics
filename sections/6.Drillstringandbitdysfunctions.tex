To improve drilling performance, a closer look must be taken at the integrity of the drill string and the borehole. Drilling dysfunctions, such as vibrations and high shock loads, can cause tool failure and hole problems, which means that to maintain the integrity of the well, drilling dysfunctions should be avoided. When looking at the consequences of drilling dysfunctions, it is easy to see how much you can optimize the overall drilling operation by using a proactive approach to either prevent or reduce these destructive forces.	

Many drilling dysfunctions with critical consequences occur because of high amplitude vibrations. The causes and effects of vibrations in the drill string will therefore be reviewed. Other drilling dysfunctions such as bit balling and interfacial severity will also be reviewed, but because they are not expected to occur in this situation, they will not be the focus of this analysis. 

\numberwithin{equation}{section}
\numberwithin{figure}{section}
\numberwithin{table}{section}


\subsection{Dysfunctions resulting from Vibrations}
The causes and effects of vibrations in the drill string are of large concern. The damage caused by vibrations can be severe both in terms of safety and of drilling efficiency, and that is the main motivation behind this analysis.
Some of the damages caused by vibrations are damage to the bit cuttings and bearings, stick/slip and stuck pipe, component and connection fatigue, and wear. This can severely damage the equipment as well as reduce the drilling rate.

Vibrations are usually more violent in a vertical well because the drill string may move more freely than in deviated wells. In addition to this, using a PDC bit accentuates transverse vibrations and bit whirl because they usually have aggressive side cutters that dig into the borehole. Vibrations are therefore expected to be one of the main sources of failure and trouble in our system. 

The restrictions on WOB related to the yield strength of the pipe mean that the aim will be to maintain a high RPM to have a good ROP. This in turn means that it will be crucial to conduct a thorough vibration analysis to optimize the drilling efficiency.

The goal of this analysis is to reduce and control the vibrations. This may be done by minimizing the generation of the vibrations and/or by reducing the life of the vibrations by damping and avoiding reflections, coupling and resonance. 


\subsubsection{Background}
There are three main types of vibrations that can occur while drilling: axial, torsional and lateral. Each of these modes has a different type of destructive nature: axial vibrations can cause bit bounce, torsional mode can cause irregular downhole rotation and lateral mode can cause large shocks as the BHA impacts the borehole wall. All three vibration types can occur during rotary drilling and they can all cause significant damage.

\paragraph{Axial Vibrations}
Axial vibrations can cause bit bounce which can damage the bit and the cutters. Axial mode is not expected to have any large impact on the drilling performance and will therefore not be focused on in this analysis.

\paragraph{Torsional Vibrations}
Stick-slip is a severe form of drill string rotational oscillation and is characterized by the release and absorption of energy. Stick-slip occurs when the drill bit cutters meet the rock and a combination of the potential of the rock and friction causes the drill bit to stay stationary for a period (stick). During stick, the bit starts to absorb energy. In the release phase (slip), the bit starts to spin out of control and the energy accumulated during the stationary period is released to the rest of the drill string as turns and twists. The severity of the torsional vibrations increases as the length of the stick period increases.

\paragraph{Lateral Vibrations}
Lateral vibrations are the most destructive type of vibrations, and are, together with axial vibrations, more violent in vertical wells.  Lateral vibration shocks are caused by the interaction between the BHA and the wellbore and may cause a high-frequency, large-magnitude, bending moment which can lead the system into whirl. 

As the system is lead into whirl, the center of rotation is offset from the center of the hole, resulting in a drill string that walks around the hole as it is rotating around itself. Whirl is a result of bit vibrations and a misalignment of the drill string and BHA. It creates a fatigue damage in the drill string through bending stress cycles which may lead to failure of connections. 

To reduce the risk of damaging the drill string through vibrations, the theory behind natural frequency will be reviewed. This will create the basis of the analysis of critical RPM.

\subsubsection{Natural Frequency}
Natural frequency is the frequency at which a system tends to oscillate in the absence of any driving or damping force. When drilling through a formation, the drilling operation leads to vibrations that pass from the bit through the BHA and finally to the drill string. If the vibrations reach the same natural frequency as that of the drilling equipment, resonance is established. 
Resonance is a phenomenon that occurs when a vibrating system drives another system to oscillate with greater amplitude at a specific frequency. In this case, it may result in a higher amplitude displacement of the equipment, thus damaging it even more. 
Rotation of the bit causes vibrations that spread through the equipment (Drillstring Vibrations and Vibration modelling, 2010). It is critical to find the rotation speed, RPM, that does not create vibrations with the same frequency as the drilling equipment’s natural frequency. By avoiding the critical RPM, the lifetime of the equipment may be increased and the drilling operations will be optimized. 

\subsection{Other Drill String and Bit Dysfunctions}

\subsubsection{Bit Balling}
Bit balling is when cuttings stick to the surface of the bit and it can happen when drilling through water reactive shale or clay formations. Electrochemical and mechanical sticking are the two main mechanisms that contribute to bit balling. If poor hydraulic design is used or the mud flow is stopped, an electrostatic force may cause cuttings to stick on the surface of the bit, and once initiated, it is easier for cuttings to build up and eventually ball up the bit. \cite{schlumvib} 

Bit balling may be detected by a sudden reduction in ROP, without any significant change in other drilling parameters. The torque is usually lower than normal since the cutters are covered up by cuttings and there may also be a sudden increase in standpipe pressure because balling reduces the annular flow area which increases the pressure. \cite{bitballing}

As soon as bit balling is spotted, the best way to mitigate the problem is to reduce the WOB and increase the flow rate. By doing this, the cuttings stuck on the drill string may be washed out \cite{bitballing}.

\subsubsection{Bottom-Hole Balling}
Bottom-hole balling is the accumulation of cuttings at the bottom of the hole which can clog the hole and prevent contact between the drill bit and the rock formation. It usually occurs when the hydrostatic pressure is very high and in impermeable rock, and results in a very high MSE and a reduced ROP. The ROP is usually unresponsive to changes in WOB, so the best way to respond is to increase the pump flow rate and the RPM.

\subsubsection{Interfacial Severity}
Interfacial severity is when a harder rock, a new layer or an inclusion in the current layer, is encountered in the formation. The force on the bit is then concentrated on the part of the bit in contact with the hard rock instead of on the entire surface area of the bit. This can cause bit damage. The best response is to reduce the WOB to minimize the damage on the bit.

